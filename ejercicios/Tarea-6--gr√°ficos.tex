% Options for packages loaded elsewhere
\PassOptionsToPackage{unicode}{hyperref}
\PassOptionsToPackage{hyphens}{url}
%
\documentclass[
]{article}
\usepackage{amsmath,amssymb}
\usepackage{lmodern}
\usepackage{ifxetex,ifluatex}
\ifnum 0\ifxetex 1\fi\ifluatex 1\fi=0 % if pdftex
  \usepackage[T1]{fontenc}
  \usepackage[utf8]{inputenc}
  \usepackage{textcomp} % provide euro and other symbols
\else % if luatex or xetex
  \usepackage{unicode-math}
  \defaultfontfeatures{Scale=MatchLowercase}
  \defaultfontfeatures[\rmfamily]{Ligatures=TeX,Scale=1}
\fi
% Use upquote if available, for straight quotes in verbatim environments
\IfFileExists{upquote.sty}{\usepackage{upquote}}{}
\IfFileExists{microtype.sty}{% use microtype if available
  \usepackage[]{microtype}
  \UseMicrotypeSet[protrusion]{basicmath} % disable protrusion for tt fonts
}{}
\makeatletter
\@ifundefined{KOMAClassName}{% if non-KOMA class
  \IfFileExists{parskip.sty}{%
    \usepackage{parskip}
  }{% else
    \setlength{\parindent}{0pt}
    \setlength{\parskip}{6pt plus 2pt minus 1pt}}
}{% if KOMA class
  \KOMAoptions{parskip=half}}
\makeatother
\usepackage{xcolor}
\IfFileExists{xurl.sty}{\usepackage{xurl}}{} % add URL line breaks if available
\IfFileExists{bookmark.sty}{\usepackage{bookmark}}{\usepackage{hyperref}}
\hypersetup{
  pdftitle={Tarea 6. Gráficos},
  pdfauthor={Paulina Luna},
  hidelinks,
  pdfcreator={LaTeX via pandoc}}
\urlstyle{same} % disable monospaced font for URLs
\usepackage[margin=1in]{geometry}
\usepackage{color}
\usepackage{fancyvrb}
\newcommand{\VerbBar}{|}
\newcommand{\VERB}{\Verb[commandchars=\\\{\}]}
\DefineVerbatimEnvironment{Highlighting}{Verbatim}{commandchars=\\\{\}}
% Add ',fontsize=\small' for more characters per line
\usepackage{framed}
\definecolor{shadecolor}{RGB}{248,248,248}
\newenvironment{Shaded}{\begin{snugshade}}{\end{snugshade}}
\newcommand{\AlertTok}[1]{\textcolor[rgb]{0.94,0.16,0.16}{#1}}
\newcommand{\AnnotationTok}[1]{\textcolor[rgb]{0.56,0.35,0.01}{\textbf{\textit{#1}}}}
\newcommand{\AttributeTok}[1]{\textcolor[rgb]{0.77,0.63,0.00}{#1}}
\newcommand{\BaseNTok}[1]{\textcolor[rgb]{0.00,0.00,0.81}{#1}}
\newcommand{\BuiltInTok}[1]{#1}
\newcommand{\CharTok}[1]{\textcolor[rgb]{0.31,0.60,0.02}{#1}}
\newcommand{\CommentTok}[1]{\textcolor[rgb]{0.56,0.35,0.01}{\textit{#1}}}
\newcommand{\CommentVarTok}[1]{\textcolor[rgb]{0.56,0.35,0.01}{\textbf{\textit{#1}}}}
\newcommand{\ConstantTok}[1]{\textcolor[rgb]{0.00,0.00,0.00}{#1}}
\newcommand{\ControlFlowTok}[1]{\textcolor[rgb]{0.13,0.29,0.53}{\textbf{#1}}}
\newcommand{\DataTypeTok}[1]{\textcolor[rgb]{0.13,0.29,0.53}{#1}}
\newcommand{\DecValTok}[1]{\textcolor[rgb]{0.00,0.00,0.81}{#1}}
\newcommand{\DocumentationTok}[1]{\textcolor[rgb]{0.56,0.35,0.01}{\textbf{\textit{#1}}}}
\newcommand{\ErrorTok}[1]{\textcolor[rgb]{0.64,0.00,0.00}{\textbf{#1}}}
\newcommand{\ExtensionTok}[1]{#1}
\newcommand{\FloatTok}[1]{\textcolor[rgb]{0.00,0.00,0.81}{#1}}
\newcommand{\FunctionTok}[1]{\textcolor[rgb]{0.00,0.00,0.00}{#1}}
\newcommand{\ImportTok}[1]{#1}
\newcommand{\InformationTok}[1]{\textcolor[rgb]{0.56,0.35,0.01}{\textbf{\textit{#1}}}}
\newcommand{\KeywordTok}[1]{\textcolor[rgb]{0.13,0.29,0.53}{\textbf{#1}}}
\newcommand{\NormalTok}[1]{#1}
\newcommand{\OperatorTok}[1]{\textcolor[rgb]{0.81,0.36,0.00}{\textbf{#1}}}
\newcommand{\OtherTok}[1]{\textcolor[rgb]{0.56,0.35,0.01}{#1}}
\newcommand{\PreprocessorTok}[1]{\textcolor[rgb]{0.56,0.35,0.01}{\textit{#1}}}
\newcommand{\RegionMarkerTok}[1]{#1}
\newcommand{\SpecialCharTok}[1]{\textcolor[rgb]{0.00,0.00,0.00}{#1}}
\newcommand{\SpecialStringTok}[1]{\textcolor[rgb]{0.31,0.60,0.02}{#1}}
\newcommand{\StringTok}[1]{\textcolor[rgb]{0.31,0.60,0.02}{#1}}
\newcommand{\VariableTok}[1]{\textcolor[rgb]{0.00,0.00,0.00}{#1}}
\newcommand{\VerbatimStringTok}[1]{\textcolor[rgb]{0.31,0.60,0.02}{#1}}
\newcommand{\WarningTok}[1]{\textcolor[rgb]{0.56,0.35,0.01}{\textbf{\textit{#1}}}}
\usepackage{graphicx}
\makeatletter
\def\maxwidth{\ifdim\Gin@nat@width>\linewidth\linewidth\else\Gin@nat@width\fi}
\def\maxheight{\ifdim\Gin@nat@height>\textheight\textheight\else\Gin@nat@height\fi}
\makeatother
% Scale images if necessary, so that they will not overflow the page
% margins by default, and it is still possible to overwrite the defaults
% using explicit options in \includegraphics[width, height, ...]{}
\setkeys{Gin}{width=\maxwidth,height=\maxheight,keepaspectratio}
% Set default figure placement to htbp
\makeatletter
\def\fps@figure{htbp}
\makeatother
\setlength{\emergencystretch}{3em} % prevent overfull lines
\providecommand{\tightlist}{%
  \setlength{\itemsep}{0pt}\setlength{\parskip}{0pt}}
\setcounter{secnumdepth}{-\maxdimen} % remove section numbering
\ifluatex
  \usepackage{selnolig}  % disable illegal ligatures
\fi

\title{Tarea 6. Gráficos}
\author{Paulina Luna}
\date{19/8/2021}

\begin{document}
\maketitle

\hypertarget{ejercicio-1}{%
\section{Ejercicio 1}\label{ejercicio-1}}

Con una sola instrucción, dibujad el gráfico de la función
y=x\^{}2−3x+30 entre −15 y 15. De título, poned ``Una parábola''. De
etiquetas, en el eje 0X poned, en formato matemático, ``x''; y en el eje
0Y, introducid \(y=x^2−3x+30\), también en formato matemático. Tenéis
que utilizar la función curve().

\begin{Shaded}
\begin{Highlighting}[]
\NormalTok{f }\OtherTok{\textless{}{-}} \ControlFlowTok{function}\NormalTok{(x)\{x}\SpecialCharTok{\^{}}\DecValTok{2} \SpecialCharTok{{-}} \DecValTok{3}\SpecialCharTok{*}\NormalTok{x}\SpecialCharTok{+}\DecValTok{30}\NormalTok{\}}
\FunctionTok{curve}\NormalTok{(f, }\AttributeTok{xlim =} \FunctionTok{c}\NormalTok{(}\SpecialCharTok{{-}}\DecValTok{15}\NormalTok{,}\DecValTok{15}\NormalTok{), }\AttributeTok{main=}\StringTok{"Una parábola"}\NormalTok{, }\AttributeTok{xlab=}\FunctionTok{expression}\NormalTok{(x), }\AttributeTok{ylab=}\FunctionTok{expression}\NormalTok{(}\AttributeTok{y=}\NormalTok{x}\SpecialCharTok{\^{}}\DecValTok{2{-}3}\SpecialCharTok{*}\NormalTok{x}\SpecialCharTok{+}\DecValTok{30}\NormalTok{), )}
\end{Highlighting}
\end{Shaded}

\includegraphics{Tarea-6--gráficos_files/figure-latex/unnamed-chunk-1-1.pdf}

\hypertarget{ejercicio-2}{%
\section{Ejercicio 2}\label{ejercicio-2}}

Considerando lo que habéis obtenido en el ejercicio anterior y siendo y
= f(x) = x\^{}2−3x+30 e I = {[}-15:15{]}, si en vez de utilizar la
función curve(), utilizamos la función plot(), ¿es correcta la sentencia
plot(f(I)) para representar la curva f en el intervalo I? En otras
palabras, dan ambas sentencias la misma gráfica? Obviamente, en la
sentencia plot(f(I)) se han omitido el resto de parámetros requeridos en
el ejercicio anterior porque no influyen para nada en la curva. Tanto si
la respuesta es afirmativa como negativa, cread la función f en R y
argumentad vuestra respuesta, considerando todos los parámetros
requeridos (título y etiquetas de ambos ejes).

\begin{Shaded}
\begin{Highlighting}[]
\NormalTok{f }\OtherTok{\textless{}{-}} \ControlFlowTok{function}\NormalTok{(x)\{x}\SpecialCharTok{\^{}}\DecValTok{2} \SpecialCharTok{{-}} \DecValTok{3}\SpecialCharTok{*}\NormalTok{x}\SpecialCharTok{+}\DecValTok{30}\NormalTok{\}}
\NormalTok{I }\OtherTok{\textless{}{-}} \FunctionTok{c}\NormalTok{(}\SpecialCharTok{{-}}\DecValTok{15}\SpecialCharTok{:}\DecValTok{15}\NormalTok{)}
\FunctionTok{plot}\NormalTok{(}\FunctionTok{f}\NormalTok{(I), }\AttributeTok{main=}\StringTok{"Una parábola"}\NormalTok{, }\AttributeTok{xlab=}\FunctionTok{expression}\NormalTok{(x), }\AttributeTok{ylab=}\FunctionTok{expression}\NormalTok{(}\AttributeTok{y=}\NormalTok{x}\SpecialCharTok{\^{}}\DecValTok{2{-}3}\SpecialCharTok{*}\NormalTok{x}\SpecialCharTok{+}\DecValTok{30}\NormalTok{))}
\end{Highlighting}
\end{Shaded}

\includegraphics{Tarea-6--gráficos_files/figure-latex/unnamed-chunk-2-1.pdf}

\hypertarget{ejercicio-3}{%
\section{Ejercicio 3}\label{ejercicio-3}}

Dibuja un gráfico semilogarítmico de la función y = 5\cdot 2\^{}x entre
-10 y 25. Utilizad la función curve(). Mostrad solo la etiqueta del eje
0Y, que ponga ``y = 5\cdot 2\^{}x'' en formato matemático.

\begin{Shaded}
\begin{Highlighting}[]
\NormalTok{f }\OtherTok{\textless{}{-}} \ControlFlowTok{function}\NormalTok{(x)\{}\DecValTok{5}\SpecialCharTok{*}\DecValTok{2}\SpecialCharTok{\^{}}\NormalTok{x\}}
\FunctionTok{curve}\NormalTok{(f, }\AttributeTok{xlim =} \FunctionTok{c}\NormalTok{(}\SpecialCharTok{{-}}\DecValTok{10}\NormalTok{,}\DecValTok{25}\NormalTok{), }\AttributeTok{ylab =} \FunctionTok{expression}\NormalTok{(}\DecValTok{5}\SpecialCharTok{*}\NormalTok{(}\DecValTok{2}\SpecialCharTok{\^{}}\NormalTok{x)))}
\end{Highlighting}
\end{Shaded}

\includegraphics{Tarea-6--gráficos_files/figure-latex/unnamed-chunk-3-1.pdf}

\hypertarget{ejercicio-4}{%
\section{Ejercicio 4}\label{ejercicio-4}}

Dibuja el gráfico de la función y\_1 = 3x utilizando la función curve().
Añade la curva y\_2=-3x, entre -10 y 20. El gráfico no debe mostrar
ninguna etiqueta. La primera curva debe ser de color azul y la segunda,
de color verde. Ponedle de título ``2 rectas'' y de subtítulo ``Dos
rectas con pendiente opuesto''. Añadid al gráfico un recuadro (con la
esquina superior izquierda en el punto (13,10)) que indique que la
función 3x es la azul y la -3x verde.

\begin{Shaded}
\begin{Highlighting}[]
\NormalTok{y1 }\OtherTok{\textless{}{-}} \ControlFlowTok{function}\NormalTok{(x)\{}\DecValTok{3}\SpecialCharTok{*}\NormalTok{x\}}
\FunctionTok{curve}\NormalTok{(y1, }\AttributeTok{col=}\StringTok{"blue"}\NormalTok{, }\AttributeTok{xlim =} \FunctionTok{c}\NormalTok{(}\SpecialCharTok{{-}}\DecValTok{10}\NormalTok{,}\DecValTok{20}\NormalTok{), }\AttributeTok{main=}\StringTok{"2 rectas"}\NormalTok{, }\AttributeTok{sub=}\StringTok{"Dos rectas con pendiente opuesto"}\NormalTok{)}
\NormalTok{y2 }\OtherTok{\textless{}{-}} \ControlFlowTok{function}\NormalTok{(x)\{}\SpecialCharTok{{-}}\DecValTok{3}\SpecialCharTok{*}\NormalTok{x\}}
\FunctionTok{curve}\NormalTok{(y2, }\AttributeTok{col=}\StringTok{"green"}\NormalTok{, }\AttributeTok{xlim =} \FunctionTok{c}\NormalTok{(}\SpecialCharTok{{-}}\DecValTok{10}\NormalTok{,}\DecValTok{20}\NormalTok{), }\AttributeTok{add=}\ConstantTok{TRUE}\NormalTok{)}
\FunctionTok{legend}\NormalTok{(}\StringTok{"topleft"}\NormalTok{, }\AttributeTok{legend=}\FunctionTok{c}\NormalTok{(}\FunctionTok{expression}\NormalTok{(}\DecValTok{3}\SpecialCharTok{*}\NormalTok{x), }\FunctionTok{expression}\NormalTok{(}\SpecialCharTok{{-}}\DecValTok{3}\SpecialCharTok{*}\NormalTok{x)),}\AttributeTok{lwd=}\DecValTok{2}\NormalTok{, }\AttributeTok{col=}\FunctionTok{c}\NormalTok{(}\StringTok{"blue"}\NormalTok{, }\StringTok{"green"}\NormalTok{), }\AttributeTok{xjust =} \DecValTok{13}\NormalTok{, }\AttributeTok{yjust =} \DecValTok{10}\NormalTok{)}
\end{Highlighting}
\end{Shaded}

\includegraphics{Tarea-6--gráficos_files/figure-latex/unnamed-chunk-4-1.pdf}

\hypertarget{ejercicio-5}{%
\section{Ejercicio 5}\label{ejercicio-5}}

Dad la instrucción que añada a un gráfico anterior la recta horizontal y
= 0 de color rojo con un grosor de 5 puntos.

\begin{Shaded}
\begin{Highlighting}[]
\NormalTok{y1 }\OtherTok{\textless{}{-}} \ControlFlowTok{function}\NormalTok{(x)\{}\DecValTok{3}\SpecialCharTok{*}\NormalTok{x\}}
\FunctionTok{curve}\NormalTok{(y1, }\AttributeTok{col=}\StringTok{"blue"}\NormalTok{, }\AttributeTok{xlim =} \FunctionTok{c}\NormalTok{(}\SpecialCharTok{{-}}\DecValTok{10}\NormalTok{,}\DecValTok{20}\NormalTok{), }\AttributeTok{main=}\StringTok{"2 rectas"}\NormalTok{, }\AttributeTok{sub=}\StringTok{"Dos rectas con pendiente opuesto"}\NormalTok{)}
\NormalTok{y2 }\OtherTok{\textless{}{-}} \ControlFlowTok{function}\NormalTok{(x)\{}\SpecialCharTok{{-}}\DecValTok{3}\SpecialCharTok{*}\NormalTok{x\}}
\FunctionTok{curve}\NormalTok{(y2, }\AttributeTok{col=}\StringTok{"green"}\NormalTok{, }\AttributeTok{xlim =} \FunctionTok{c}\NormalTok{(}\SpecialCharTok{{-}}\DecValTok{10}\NormalTok{,}\DecValTok{20}\NormalTok{), }\AttributeTok{add=}\ConstantTok{TRUE}\NormalTok{)}
\FunctionTok{legend}\NormalTok{(}\StringTok{"topleft"}\NormalTok{, }\AttributeTok{legend=}\FunctionTok{c}\NormalTok{(}\FunctionTok{expression}\NormalTok{(}\DecValTok{3}\SpecialCharTok{*}\NormalTok{x), }\FunctionTok{expression}\NormalTok{(}\SpecialCharTok{{-}}\DecValTok{3}\SpecialCharTok{*}\NormalTok{x)),}\AttributeTok{lwd=}\DecValTok{2}\NormalTok{, }\AttributeTok{col=}\FunctionTok{c}\NormalTok{(}\StringTok{"blue"}\NormalTok{, }\StringTok{"green"}\NormalTok{), }\AttributeTok{xjust =} \DecValTok{13}\NormalTok{, }\AttributeTok{yjust =} \DecValTok{10}\NormalTok{)}
\FunctionTok{abline}\NormalTok{(}\AttributeTok{h=}\DecValTok{0}\NormalTok{, }\AttributeTok{col=}\StringTok{"red"}\NormalTok{, }\AttributeTok{lwd=}\DecValTok{5}\NormalTok{)}
\end{Highlighting}
\end{Shaded}

\includegraphics{Tarea-6--gráficos_files/figure-latex/unnamed-chunk-5-1.pdf}

\hypertarget{ejercicio-6}{%
\section{Ejercicio 6}\label{ejercicio-6}}

Dad la instrucción que añada a un gráfico anterior la recta y = 2x+7 de
color azul con un grosor de 2 puntos.

\begin{Shaded}
\begin{Highlighting}[]
\NormalTok{y1 }\OtherTok{\textless{}{-}} \ControlFlowTok{function}\NormalTok{(x)\{}\DecValTok{3}\SpecialCharTok{*}\NormalTok{x\}}
\FunctionTok{curve}\NormalTok{(y1, }\AttributeTok{col=}\StringTok{"blue"}\NormalTok{, }\AttributeTok{xlim =} \FunctionTok{c}\NormalTok{(}\SpecialCharTok{{-}}\DecValTok{10}\NormalTok{,}\DecValTok{20}\NormalTok{), }\AttributeTok{main=}\StringTok{"2 rectas"}\NormalTok{, }\AttributeTok{sub=}\StringTok{"Dos rectas con pendiente opuesto"}\NormalTok{)}
\NormalTok{y2 }\OtherTok{\textless{}{-}} \ControlFlowTok{function}\NormalTok{(x)\{}\SpecialCharTok{{-}}\DecValTok{3}\SpecialCharTok{*}\NormalTok{x\}}
\FunctionTok{curve}\NormalTok{(y2, }\AttributeTok{col=}\StringTok{"green"}\NormalTok{, }\AttributeTok{xlim =} \FunctionTok{c}\NormalTok{(}\SpecialCharTok{{-}}\DecValTok{10}\NormalTok{,}\DecValTok{20}\NormalTok{), }\AttributeTok{add=}\ConstantTok{TRUE}\NormalTok{)}
\FunctionTok{legend}\NormalTok{(}\StringTok{"topleft"}\NormalTok{, }\AttributeTok{legend=}\FunctionTok{c}\NormalTok{(}\FunctionTok{expression}\NormalTok{(}\DecValTok{3}\SpecialCharTok{*}\NormalTok{x), }\FunctionTok{expression}\NormalTok{(}\SpecialCharTok{{-}}\DecValTok{3}\SpecialCharTok{*}\NormalTok{x)),}\AttributeTok{lwd=}\DecValTok{2}\NormalTok{, }\AttributeTok{col=}\FunctionTok{c}\NormalTok{(}\StringTok{"blue"}\NormalTok{, }\StringTok{"green"}\NormalTok{), }\AttributeTok{xjust =} \DecValTok{13}\NormalTok{, }\AttributeTok{yjust =} \DecValTok{10}\NormalTok{)}
\FunctionTok{abline}\NormalTok{(}\AttributeTok{h=}\DecValTok{0}\NormalTok{, }\AttributeTok{col=}\StringTok{"red"}\NormalTok{, }\AttributeTok{lwd=}\DecValTok{5}\NormalTok{)}
\FunctionTok{abline}\NormalTok{(}\DecValTok{2}\NormalTok{,}\DecValTok{7}\NormalTok{, }\AttributeTok{col=}\StringTok{"blue"}\NormalTok{, }\AttributeTok{lwd=}\DecValTok{2}\NormalTok{)}
\end{Highlighting}
\end{Shaded}

\includegraphics{Tarea-6--gráficos_files/figure-latex/unnamed-chunk-6-1.pdf}

\end{document}
